\documentclass[a4paper, 11pt]{article}

  \usepackage[utf8]{inputenc}
  \usepackage[czech]{babel}
  \usepackage{times}
  \usepackage[unicode]{hyperref}
  \usepackage[total={17cm,24cm}, top=3cm, left=2cm]{geometry}
  \usepackage{graphics}
 


  
  
  %--------------------------------------------
  \begin{document}
  
  \begin{titlepage}
    \begin{center} 
    \textsc{\Huge Vysoké učení technické v~Brně\\ \huge Fakulta informačních technologií \\}
    \vspace{\stretch{0.382}}
    \LARGE Typografie a publikování\,--\ 3. projekt \\ \Huge Tabulky a obrázky
    \vspace{\stretch{0.618}}
    \end{center}
    \Large \today \hfill Matúš Juštik

  \end{titlepage}

\section{\LaTeX}
\subsection{Ako funguje}
\LaTeX \ je značkovacím jazykem, který je založen na jádru programu \TeX. Ten poskytuje
v~základu primitivní sázecí operace. Z~nich jsou skládány složitější bloky příkazů neboli
takzvaná makra. Autoři vytvořili nejjednodušší formát pro tvorbu dokumentu - PlainTex.
Z~něj vychází všechny ostatní formáty a při překladu se pro něj používá jiných kompilátorů.
Holá kostra umožňuje plnou kontrolu nad vlastnostmi a sazbou textu, ale je nutná hlubší
znalost celého programovacího jazyka. \cite{FITMT13353}

\subsection{Čo je \LaTeX}
Systém \LaTeX\ je balík maker pre DTP systém \TeX. Nie je sádzajúcim programom, ale jeho nadstavbou. \par
Teda uľachčuje používanie \TeX u, ktorý zhrnie spracovanie dokomuentu ako takého a jeho prvky tak, aby
čo najviac zrýchlili písanie inak zložitých a zdlhavých postupou, ktoré \TeX poskytuje. \cite{FITMT13471}

\subsection{Divná výslovnosť?}
Výslovnosť sa vie stretnúť s~neporozumením, keďže latex je guma a \LaTeX je  niečo iné. Teda \LaTeX sa číta následovne [latech] alebo len [tech], a latex [latex]. Meno vzniklo aj po tvorcovi \LaTeX u Leslie Lamport-ovi, "Lamportův TeX". \cite{Martinek}

\subsection{Bibliografia}
Bibliografia je podstatna časť \LaTeX pri vedeckych publikáciach. \cite{kopka1995guide}
Program bibtex je nástroj, ktorý vygenerované \TeX príkazy zahrnie do \LaTeX documentu na
znázornenie/zobrazenie zoznamu preferencií. \cite{thummala2007bibliography}

\section{Informatika-Arduino}
\subsection{Čo je Arduino}
Arduine je mikrokontrolér, teda je to malý počítač a na jednom zapojenom obehu obsahuje jadro procesora, pamäť a proprogramovateľné input/output periféria. \cite{smith2011introduction}

\subsection{Jayzk C}
Štandard C++, ktorý sa používa pri práci s~arduino je mierne poupravený. \par Každopádne preberá
všetky štandardné knižnice z~C ako tak aj svoje vlastné. Urobia rovnakú prácu aj v~C aj C++.
\cite{eckel2000myslime} 

\subsection{Funkcie}
Funkcie z~C++ fungujú rovnako ako na každom jednom progeamovacom jazyku. Je to akoby taký program v~programe. Funkciu, ktorú definujete, môžeme zavolať odkiaľkoľvek v~programe, kde sa nachádzajú aj premenné. \cite{monk2016programming}
	
	\newpage
	\bibliographystyle{czechiso}
	\bibliography{proj4}


\end{document}
